%%%%%%%%%%%%%%%%%%%%%%%%%%%%%%%%%%%%%%%%%
% Large Colored Title Article
% LaTeX Template
% Version 1.1 (25/11/12)
%
% This template has been downloaded from:
% http://www.LaTeXTemplates.com
%
% Original author:
% Frits Wenneker (http://www.howtotex.com)
%
% License:
% CC BY-NC-SA 3.0 (http://creativecommons.org/licenses/by-nc-sa/3.0/)
%
%%%%%%%%%%%%%%%%%%%%%%%%%%%%%%%%%%%%%%%%%

%----------------------------------------------------------------------------------------
%	PACKAGES AND OTHER DOCUMENT CONFIGURATIONS
%----------------------------------------------------------------------------------------

\documentclass[DIV=calc, paper=a4, fontsize=11pt, twocolumn]{scrartcl}	 % A4 paper and 11pt font size

\usepackage{attachfile}
\usepackage{lipsum} % Used for inserting dummy 'Lorem ipsum' text into the template
\usepackage[english]{babel} % English language/hyphenation
\usepackage[protrusion=true,expansion=true]{microtype} % Better typography
\usepackage{amsmath,amsfonts,amsthm} % Math packages
\usepackage[svgnames]{xcolor} % Enabling colors by their 'svgnames'
\usepackage[hang, small,labelfont=bf,up,textfont=it,up]{caption} % Custom captions under/above floats in tables or figures
\usepackage{booktabs} % Horizontal rules in tables
\usepackage{fix-cm}	 % Custom font sizes - used for the initial letter in the document

\usepackage{sectsty} % Enables custom section titles
\allsectionsfont{\usefont{OT1}{phv}{b}{n}} % Change the font of all section commands

\usepackage{fancyhdr} % Needed to define custom headers/footers
\pagestyle{fancy} % Enables the custom headers/footers
\usepackage{lastpage} % Used to determine the number of pages in the document (for "Page X of Total")

% Headers - all currently empty
\lhead{}
\chead{}
\rhead{}

% Footers
\lfoot{}
\cfoot{}
\rfoot{\footnotesize Page \thepage\ of \pageref{LastPage}} % "Page 1 of 2"

\renewcommand{\headrulewidth}{0.0pt} % No header rule
\renewcommand{\footrulewidth}{0.4pt} % Thin footer rule

\usepackage{lettrine} % Package to accentuate the first letter of the text
\newcommand{\initial}[1]{ % Defines the command and style for the first letter
\lettrine[lines=3,lhang=0.3,nindent=0em]{
\color{DarkGoldenrod}
{\textsf{#1}}}{}}

%----------------------------------------------------------------------------------------
%	TITLE SECTION
%----------------------------------------------------------------------------------------

\usepackage{titling} % Allows custom title configuration

\newcommand{\HorRule}{\color{DarkGoldenrod} \rule{\linewidth}{1pt}} % Defines the gold horizontal rule around the title

\pretitle{\vspace{-30pt} \begin{flushleft} \HorRule \fontsize{50}{50} \usefont{OT1}{phv}{b}{n} \color{DarkRed} \selectfont} % Horizontal rule before the title

\title{Build a Fortran project through CMake} % Your article title

\posttitle{\par\end{flushleft}\vskip 0.5em} % Whitespace under the title

\preauthor{\begin{flushleft}\large \lineskip 0.5em \usefont{OT1}{phv}{b}{sl} \color{DarkRed}} % Author font configuration

\author{J.Y.Li} % Your name

\postauthor{\footnotesize \usefont{OT1}{phv}{m}{sl} \color{Black} % Configuration for the institution name
Dalian Neusoft University of Information % Your institution

\par\end{flushleft}\HorRule} % Horizontal rule after the title

\date{} % Add a date here if you would like one to appear underneath the title block

%----------------------------------------------------------------------------------------

\begin{document}

\maketitle % Print the title

\thispagestyle{fancy} % Enabling the custom headers/footers for the first page 

%----------------------------------------------------------------------------------------
%	ABSTRACT
%----------------------------------------------------------------------------------------

% The first character should be within \initial{}
\initial{I}\textbf{n recent days I made a hard work for a more flexible Fortran project build. Using this template you will made a possible build Fortran project in Linux environment and made a data analysis program at the same time. The project in this PDF file append. \attachfile{Cmake-Fortran.tar.gz} }

%----------------------------------------------------------------------------------------
%	ARTICLE CONTENTS
%----------------------------------------------------------------------------------------

\section*{Introduction}

In recently work we have to compute many version Fortran program.
These programs all developed in VS2010. 
When transform source code to Linux server and compute target data.
We got a trouble that analysis dependence of code and library is very hard.
So we need make a lighter faster method to build the project.

In last BLOG we have make a method use in C/C++. 
Now we even choose this tool to build Fortran.
To make an easily faster auto dependence invokes the compiler and also makes a huge data analysis tool at the same time.
In this project we make a multiple target project (two executable and one library). 
The module core is the target build by Fortran program.
And the other executable is data analysis program. 
The library writing for diagram output dependence on CERN ROOT.

\section{How to use this template project}

At very first you need get the apped of the template project.
Then UNTAR the package.
Then you get the template project.
You may add you projecto files the the path \textbf{Cmake-Fortran/src/core}
And then you need to change the \textbf{CMakeLists.txt} file in compare to the core directory.
Then I will try to describe the CMakeLists.txt change step by step
\begin{itemize}
\item Set the project name in \textbf{project(your project name)}
\item Set compiler you need compiler the program. MPIIFORT is default. \\
	\textbf{set (CMAKE\_Fortran\_COMPILER mpiifort)}
\item Optimaze the compiler parameter in syntax\\
	for release \\
	\textbf{set (CMAKE\_Fortran\_FLAGS\_RELEASE "-O3")}\\
	for debug \\
	\textbf{set (CMAKE\_Fortran\_FLAGS\_DEBUG   "-O0 -g")}
\item Adject the include directory.  in syntax \\
	\textbf{include\_directories( }\\
	We need have a graphics library in depandence by default.\\
	Special explain in this syntax. you need add the floader line by line.
	Don't forget there are even have a ")" at the tail of this syntax.
\item Add source code file in syntax \\
	\textbf{set (src }\\
	The format is the same as add include foader.\\
	At the capital of the file name you need add a string before it.
	\textbf{ \$\{CMAKE\_SOURCE\_DIR\}/core/ }
\end{itemize}

That is mainly you need to de change the CMakeLists.txt.

\section{How to use data analysis module}

In this section you need have many C/C++ develop experience.
And the you can use CERN ROOT freely.
This module have a good feature that you can analysis huge count of data.
In foader MakePlot have only one file.
Edit this file read data file and get the data array send it to library event.
Edit a ROOT script and change it to the CERN module by CINT.

\section{build project and import the project to project}

In floader script have a build script.
Run it and you will configure the project and compile the project. 
And then open floader \textbf{build\\bin}then you will see a file name MTF4.exe
that is you main target of you fortran project.
open your eclipse(you need install CDT) click file $\rightarrow$ import $\rightarrow$ choose General/Existing projects into weorkspace
 $\rightarrow$select root directory(\textbf{the path is you project built directory})  $\rightarrow$ finish
then you will see a new project in you eclipse \textbf{project explorer}

\end{document}
